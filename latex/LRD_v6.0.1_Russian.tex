\documentclass[12pt,a4paper]{article}

% ===== PACKAGES =====
\usepackage{fontspec}
\usepackage{polyglossia}
\setdefaultlanguage{russian}
\setotherlanguage{english}
\setmainfont{DejaVu Serif}
\setsansfont{DejaVu Sans}
\setmonofont{DejaVu Sans Mono}
\usepackage{amsmath,amssymb,amsthm}
\usepackage{mathtools}
\usepackage{graphicx}
\usepackage{booktabs}
\usepackage{array}
\usepackage{multirow}
\usepackage{longtable}
\usepackage{hyperref}
\usepackage[margin=1in]{geometry}
\usepackage{setspace}
\usepackage{enumitem}
\usepackage{caption}
\usepackage{fancyhdr}
\usepackage{xcolor}
\usepackage{tcolorbox}

% ===== THEOREM ENVIRONMENTS =====
\newtheorem{theorem}{Теорема}[section]
\newtheorem{lemma}[theorem]{Лемма}
\newtheorem{proposition}[theorem]{Утверждение}
\newtheorem{corollary}[theorem]{Следствие}
\newtheorem{definition}[theorem]{Определение}
\newtheorem{hypothesis}{Гипотеза}
\newtheorem{result}{Результат}

\theoremstyle{remark}
\newtheorem{remark}[theorem]{Замечание}
\newtheorem{example}[theorem]{Пример}

% ===== CUSTOM COMMANDS =====
\newcommand{\E}{\mathbb{E}}
\newcommand{\R}{\mathbb{R}}
\newcommand{\N}{\mathbb{N}}
\newcommand{\Prob}{\mathbb{P}}
\newcommand{\Var}{\mathrm{Var}}
\newcommand{\Cov}{\mathrm{Cov}}
\newcommand{\dd}{\mathrm{d}}

% ===== DOCUMENT INFO =====
\title{%
\textbf{Долговременная память и микроструктурная универсальность:\\
От закона квадратного корня к фрактальной волатильности}\\[1em]
\large LRD v6.0.1: Воспроизводимая система аудита с теоретическим обоснованием
}

\author{%
Игорь Чечельницкий\thanks{Независимый исследователь, Ашкелон, Израиль. ORCID: 0009-0007-4607-1946}
}

\date{Декабрь 2025\\[0.5em]\small Версия 6.0.1 --- Zenodo DOI: 10.5281/zenodo.17993213}

% ===== HEADER/FOOTER =====
\pagestyle{fancy}
\fancyhf{}
\fancyhead[L]{\small LRD v6.0.1}
\fancyhead[R]{\small Чечельницкий (2025)}
\fancyfoot[C]{\thepage}
\renewcommand{\headrulewidth}{0.4pt}

\begin{document}

\maketitle

\begin{abstract}
\noindent
Представлена версия LRD v6.0.1 --- воспроизводимая система для обнаружения и валидации долговременной зависимости (LRD) в сложных временных рядах с теоретическим обоснованием, связывающим микроструктурную универсальность с макроскопической фрактальной памятью. Система объединяет оценку масштабирования DFA-2, блочные бутстрап-доверительные интервалы и тесты фальсификации на фазово-рандомизированных суррогатах. Эмпирические данные из четырёх областей --- волатильность Bitcoin, последовательности землетрясений, вариабельность сердечного ритма (HRV) и геномная ДНК --- демонстрируют персистентную память ($H > 0.5$) как универсальное свойство сложных систем.

Ключевой теоретический вклад --- интеграция недавних высокоточных результатов по закону квадратного корня (SRL) ценового импакта: Sato и Kanazawa (\textit{Phys. Rev. Lett.} \textbf{135}, 257401, 2025) установили, что рыночный импакт масштабируется как $\Delta p \propto Q^{\delta}$ с $\delta \approx 1/2$ универсально для всех ликвидных акций и трейдеров на Токийской фондовой бирже. Мы показываем, что эта микроструктурная универсальность обеспечивает механистический якорь для наблюдаемой макроскопической LRD в волатильности: стратегии дробления ордеров крупных трейдеров в сочетании с универсальным масштабированием импакта генерируют персистентные корреляции в потоке ордеров, проявляющиеся как фрактальная кластеризация волатильности.

\medskip
\noindent\textbf{Ключевые слова:} долговременная зависимость; фрактальная память; показатель Хёрста; DFA; закон квадратного корня; микроструктура рынка; Bitcoin; кластеризация волатильности; эконофизика; универсальность
\end{abstract}

\newpage
\tableofcontents
\newpage

%==============================================================================
\section{Введение}
\label{sec:introduction}
%==============================================================================

Обнаружение и характеризация долговременной зависимости (Long-Range Dependence, LRD) в эмпирических временных рядах стало центральной проблемой в дисциплинах от геофизики до финансов. Процесс $\{X_t\}$ демонстрирует LRD, когда его автокорреляционная функция $\rho(k)$ убывает гиперболически, а не экспоненциально:
\begin{equation}
\rho(k) \sim c \cdot k^{2H-2}, \quad k \to \infty, \quad H \in (0.5, 1),
\label{eq:lrd_def}
\end{equation}
где $H$ --- показатель Хёрста, $c > 0$ --- константа. Это медленное убывание означает, что шоки распространяются по всем временным масштабам, создавая <<память>>, которую стандартные модели с короткой памятью (ARMA, GARCH) не способны уловить.

\subsection{Разрыв микро-макро}

Несмотря на обширную эмпирическую документацию, фундаментальный вопрос оставался открытым: \emph{какой микроскопический механизм порождает макроскопическую фрактальную память?} Для финансовых рынков ответ включает взаимодействие:
\begin{enumerate}[label=(\roman*)]
    \item \textbf{Дробление ордеров}: Крупные институциональные инвесторы исполняют метаордера, разбивая их на множество мелких дочерних ордеров в течение длительного периода для минимизации рыночного импакта.
    \item \textbf{Масштабирование импакта}: Каждый дочерний ордер двигает цену согласно универсальному закону --- закону квадратного корня (SRL).
    \item \textbf{Агрегированная персистентность}: Кумулятивный эффект генерирует долговременные корреляции в потоке ордеров и, следовательно, в волатильности.
\end{enumerate}

\subsection{Вклад данной работы}

LRD v6.0.1 вносит следующий вклад:

\begin{enumerate}
    \item \textbf{Эмпирическая система аудита}: Воспроизводимый пайплайн для обнаружения LRD с использованием DFA-2, блочного бутстрапа и суррогатных тестов фальсификации, применённый к Bitcoin, землетрясениям, HRV и геномике.
    
    \item \textbf{Теоретическое обоснование}: Интеграция результата универсальности SRL Sato--Kanazawa (\textit{Phys. Rev. Lett.} 2025) как микроструктурного якоря для макроскопической LRD.
    
    \item \textbf{Фальсифицируемость}: Явные условия, при которых заявления об LRD должны быть отвергнуты, отличая подлинную память от ложного масштабирования.
    
    \item \textbf{Релевантность для ИИ}: Импликации для моделей машинного обучения, работающих на рынках с фрактальной памятью, где предположения i.i.d. нарушены.
\end{enumerate}

%==============================================================================
\section{Теоретические основы}
\label{sec:theory}
%==============================================================================

\subsection{Долговременная зависимость: определения}

\begin{definition}[Долговременная зависимость]
Стационарный процесс $\{X_t\}_{t \in \mathbb{Z}}$ с автоковариацией $\gamma(k) = \Cov(X_t, X_{t+k})$ демонстрирует \emph{долговременную зависимость}, если:
\begin{equation}
\sum_{k=-\infty}^{\infty} |\gamma(k)| = \infty.
\end{equation}
Эквивалентно, спектральная плотность $f(\lambda)$ удовлетворяет $f(\lambda) \sim c_f |\lambda|^{1-2H}$ при $\lambda \to 0$ для некоторого $H \in (0.5, 1)$.
\end{definition}

Показатель Хёрста $H$ служит каноническим параметром:
\begin{itemize}
    \item $H = 0.5$: Отсутствие памяти (белый шум)
    \item $H > 0.5$: Персистентная/положительная LRD (тренды продолжаются)
    \item $H < 0.5$: Антиперсистентная/отрицательная LRD (возврат к среднему)
\end{itemize}

\subsection{Закон квадратного корня ценового импакта}

Закон квадратного корня (Square-Root Law, SRL) описывает, как рыночные цены реагируют на объём торгов:

\begin{hypothesis}[Универсальность SRL]
\label{hyp:srl}
Для любого ликвидного финансового рынка ожидаемый ценовой импакт $I(Q)$ исполнения метаордера объёма $Q$ удовлетворяет:
\begin{equation}
I(Q) \propto \sigma \sqrt{\frac{Q}{V}},
\label{eq:srl}
\end{equation}
где $\sigma$ --- дневная волатильность, $V$ --- дневной объём. Показатель $\delta = 1/2$ универсален --- не зависит от идентичности акции, трейдера или торговой площадки.
\end{hypothesis}

\subsubsection{Эмпирическое подтверждение: Sato и Kanazawa (2025)}

Универсальность SRL теперь строго подтверждена. Sato и Kanazawa проанализировали восемь лет (2012--2019) полных данных об ордерах Токийской фондовой биржи:
\begin{itemize}
    \item Все ордера, сделки и отмены
    \item Все индивидуальные торговые аккаунты (анонимизированные)
    \item Реконструкция метаордеров по всем ликвидным акциям
\end{itemize}

Их ключевой результат:

\begin{result}[Универсальность SRL --- Sato и Kanazawa 2025]
Показатель ценового импакта $\delta$ равен $0.5 \pm 0.05$ для \emph{каждой} акции и \emph{каждой} категории трейдеров на TSE, без систематических отклонений. Закон квадратного корня является строгим универсальным масштабированием.
\end{result}

Этот результат опубликован в \textit{Physical Review Letters} и обсуждён в APS Physics Viewpoint Bouchaud.

\subsection{От микроструктуры к макроскопической памяти}

Связь между SRL и LRD работает через следующий механизм:

\begin{proposition}[Связь микро-макро]
\label{prop:micro_macro}
При следующих условиях:
\begin{enumerate}[label=(\alph*)]
    \item Крупные трейдеры дробят метаордера на дочерние ордера, исполняемые во времени
    \item Каждый дочерний ордер воздействует на цену согласно SRL с $\delta = 1/2$
    \item Трейдеры оптимизируют исполнение для минимизации общей стоимости импакта
\end{enumerate}
Результирующий процесс потока ордеров $\{q_t\}$ демонстрирует долговременную зависимость в своей последовательности знаков, и процесс волатильности $\{\sigma_t^2\}$ наследует эту персистентность.
\end{proposition}

%==============================================================================
\section{Методология: протокол аудита LRD}
\label{sec:methodology}
%==============================================================================

\subsection{Обзор}

Протокол аудита LRD v6 состоит из четырёх обязательных этапов:

\begin{enumerate}
    \item \textbf{Оценка масштабирования}: DFA-2 для оценки показателя Хёрста $H$
    \item \textbf{Квантификация неопределённости}: Блочные бутстрап-доверительные интервалы
    \item \textbf{Тестирование фальсификации}: Фазово-рандомизированные суррогаты
    \item \textbf{Анализ чувствительности}: Проверка робастности по диапазону масштабов
\end{enumerate}

Если любой этап не пройден, заявление об LRD отвергается.

\subsection{Детрендированный флуктуационный анализ (DFA-2)}

DFA был введён Peng et al. для обнаружения LRD в нестационарных последовательностях со встроенными трендами.

\subsubsection{Алгоритм}

Для временного ряда $\{x_i\}_{i=1}^N$:

\begin{enumerate}
    \item \textbf{Интегрирование}: Вычислить кумулятивную сумму $y(k) = \sum_{i=1}^k (x_i - \bar{x})$
    
    \item \textbf{Сегментация}: Разделить $\{y(k)\}$ на непересекающиеся сегменты длины $n$
    
    \item \textbf{Детрендирование}: В каждом сегменте подогнать полином $p_\nu(k)$ порядка $m$ (DFA-$m$ использует степень $m$; мы используем $m=2$)
    
    \item \textbf{Флуктуация}: Вычислить среднеквадратичную флуктуацию:
    \begin{equation}
    F(n) = \sqrt{\frac{1}{N}\sum_{k=1}^N \left[y(k) - p_\nu(k)\right]^2}
    \end{equation}
    
    \item \textbf{Масштабирование}: Повторить для различных $n$; соотношение
    \begin{equation}
    F(n) \propto n^H
    \label{eq:dfa_scaling}
    \end{equation}
    даёт $H$ из наклона в логарифмических координатах.
\end{enumerate}

\begin{tcolorbox}[title=Правило принятия решения при аудите]
\textbf{Принять заявление об LRD} тогда и только тогда, когда:
\begin{enumerate}
    \item DFA-2 даёт $\hat{H} > 0.5$ с бутстрап-ДИ полностью выше $0.5$
    \item Фазово-рандомизированные суррогаты дают значительно более низкий $H$
    \item Масштабирование сохраняется по нескольким диапазонам масштабов
\end{enumerate}
В противном случае \textbf{отвергнуть} заявление об LRD.
\end{tcolorbox}

%==============================================================================
\section{Эмпирические результаты}
\label{sec:results}
%==============================================================================

Мы применяем протокол аудита LRD к четырём различным областям.

\subsection{Область 1: Волатильность Bitcoin}

\begin{table}[htbp]
\centering
\caption{Результаты DFA-2 для временных рядов Bitcoin}
\label{tab:bitcoin_results}
\begin{tabular}{@{}lccc@{}}
\toprule
\textbf{Ряд} & \textbf{$\hat{H}$} & \textbf{95\% ДИ} & \textbf{$p$-значение суррогата} \\
\midrule
Лог-доходности $r_t$ & 0.52 & [0.49, 0.55] & 0.34 \\
Абсолютные доходности $|r_t|$ & 0.71 & [0.68, 0.74] & $< 0.001$ \\
Объём торгов $V_t$ & 0.93 & [0.89, 0.96] & $< 0.001$ \\
\bottomrule
\end{tabular}
\end{table}

\begin{result}[LRD в Bitcoin]
Сырые доходности Bitcoin не показывают значимой LRD ($H \approx 0.5$), что согласуется со слабой формой рыночной эффективности. Однако \emph{прокси волатильности} демонстрируют сильную персистентную LRD ($H \approx 0.71$), а объём торгов показывает очень высокую персистентность ($H \approx 0.93$). Эти результаты проходят тесты фальсификации на суррогатах.
\end{result}

\subsection{Область 2: Последовательности землетрясений}

\begin{table}[htbp]
\centering
\caption{Результаты DFA-2 для сейсмических временных рядов}
\label{tab:earthquake_results}
\begin{tabular}{@{}lccc@{}}
\toprule
\textbf{Ряд} & \textbf{$\hat{H}$} & \textbf{95\% ДИ} & \textbf{$p$-значение суррогата} \\
\midrule
Кумулятивный момент (Италия) & 0.87 & [0.83, 0.91] & $< 0.001$ \\
Кумулятивный момент (Глобальный) & 0.84 & [0.80, 0.88] & $< 0.001$ \\
Межсобытийные интервалы & 0.68 & [0.64, 0.72] & $< 0.01$ \\
\bottomrule
\end{tabular}
\end{table}

\begin{result}[LRD в сейсмике]
Последовательности землетрясений демонстрируют сильную LRD с $H \approx 0.85$--$0.87$, указывая, что периоды высокой сейсмической активности склонны кластеризоваться.
\end{result}

\subsection{Область 3: Вариабельность сердечного ритма (HRV)}

\begin{table}[htbp]
\centering
\caption{Результаты DFA-2 для HRV (здоровые субъекты)}
\label{tab:hrv_results}
\begin{tabular}{@{}lccc@{}}
\toprule
\textbf{Мера} & \textbf{$\hat{H}$} & \textbf{95\% ДИ} & \textbf{$p$-значение суррогата} \\
\midrule
RR-интервалы (покой) & 0.92 & [0.88, 0.96] & $< 0.001$ \\
RR-интервалы (активность) & 0.78 & [0.74, 0.82] & $< 0.001$ \\
\bottomrule
\end{tabular}
\end{table}

\begin{result}[LRD в HRV]
Вариабельность сердечного ритма здоровых людей демонстрирует сильную LRD ($H \approx 0.9$ в покое). Эта фрактальная структура --- маркер здоровой кардиодинамики.
\end{result}

\subsection{Область 4: Геномные последовательности ДНК}

\begin{table}[htbp]
\centering
\caption{Результаты DFA для последовательностей ДНК}
\label{tab:dna_results}
\begin{tabular}{@{}lcc@{}}
\toprule
\textbf{Тип последовательности} & \textbf{$\hat{H}$} & \textbf{LRD присутствует?} \\
\midrule
Интроны (некодирующие) & $0.60$--$0.65$ & Да \\
Межгенные области & $0.62$--$0.68$ & Да \\
Экзоны (кодирующие) & $0.50$--$0.52$ & Нет \\
кДНК (процессированная мРНК) & $0.48$--$0.51$ & Нет \\
\bottomrule
\end{tabular}
\end{table}

\begin{result}[LRD в геноме]
Некодирующие области ДНК (интроны, межгенные) демонстрируют значимую LRD ($H \approx 0.65$), тогда как кодирующие области (экзоны) --- нет. Это подтверждает оригинальные находки Peng et al. и предполагает, что фрактальная организация в геномах служит регуляторным или эволюционным функциям.
\end{result}

\subsection{Сводка эмпирических результатов}

\begin{table}[htbp]
\centering
\caption{Кросс-доменная сводка доказательств LRD}
\label{tab:summary}
\begin{tabular}{@{}llcc@{}}
\toprule
\textbf{Область} & \textbf{Наблюдаемая} & \textbf{$\hat{H}$} & \textbf{LRD подтверждена} \\
\midrule
Финансы (Bitcoin) & Прокси волатильности & 0.71 & $\checkmark$ \\
Финансы (Bitcoin) & Объём торгов & 0.93 & $\checkmark$ \\
Геофизика & Сейсмический момент & 0.85 & $\checkmark$ \\
Физиология & HRV (здоровые) & 0.92 & $\checkmark$ \\
Геномика & Некодирующая ДНК & 0.65 & $\checkmark$ \\
\bottomrule
\end{tabular}
\end{table}

%==============================================================================
\section{Обсуждение}
\label{sec:discussion}
%==============================================================================

\subsection{Универсальность фрактальной памяти}

Присутствие LRD в волатильности Bitcoin, последовательностях землетрясений, кардиодинамике и геномной ДНК предполагает, что фрактальная память --- не курьёз отдельных систем, а \emph{универсальное свойство} сложных систем с:
\begin{itemize}
    \item Множеством взаимодействующих компонентов
    \item Нелинейной динамикой
    \item Иерархической организацией
\end{itemize}

\subsection{Микроструктурное обоснование финансовой LRD}

Для финансовых рынков мы теперь имеем полное объяснение микро-макро:

\begin{enumerate}
    \item \textbf{Микро}: Универсальный SRL ($\delta = 1/2$) управляет импактом отдельной сделки
    \item \textbf{Мезо}: Оптимальное исполнение $\to$ дробление ордеров $\to$ коррелированный поток ордеров
    \item \textbf{Макро}: Долговременная зависимость в прокси волатильности
\end{enumerate}

Результат Sato--Kanazawa закрывает эмпирический разрыв, демонстрируя, что SRL выполняется универсально.

\subsection{Импликации для ИИ и машинного обучения}

\subsubsection{Спецификация модели}

Стандартные подходы машинного обучения часто предполагают i.i.d. остатки или зависимость с короткой памятью. При наличии LRD эти предположения нарушены, что ведёт к:
\begin{itemize}
    \item Недооценке неопределённости прогнозов
    \item Ложному обнаружению паттернов (переобучение на персистентных флуктуациях)
    \item Неверным доверительным интервалам для метрик риска
\end{itemize}

\subsubsection{Рекомендуемые подходы}

\begin{enumerate}
    \item \textbf{Инженерия признаков}: Включить показатели Хёрста, полученные через DFA, как предиктивные признаки
    \item \textbf{Архитектура модели}: Использовать архитектуры с длинным контекстом (Трансформеры) или явной памятью (LSTM с длинным состоянием)
    \item \textbf{Функции потерь}: Адаптировать функции потерь для учёта коррелированных ошибок
    \item \textbf{Квантификация неопределённости}: Использовать блочный бутстрап или другие методы, учитывающие LRD
\end{enumerate}

\subsubsection{Применения к криптовалютам}

Bitcoin и другие криптовалюты показывают особенно сильную LRD в объёме и волатильности. Торговые алгоритмы и системы управления рисками должны:
\begin{itemize}
    \item Ожидать кластеризацию волатильности за пределами затухания GARCH-типа
    \item Корректировать размер позиций для персистентных режимов
    \item Мониторить локальные оценки $H$ для обнаружения смены режима
\end{itemize}

%==============================================================================
\section{Заключение}
\label{sec:conclusion}
%==============================================================================

LRD v6.0.1 предоставляет воспроизводимую, фальсифицируемую систему для обнаружения и интерпретации долговременной зависимости в сложных временных рядах. Ключевые достижения:

\begin{enumerate}
    \item \textbf{Эмпирические}: Подтверждение LRD в четырёх областях со строгим аудитом (DFA-2 + бутстрап + суррогаты)
    
    \item \textbf{Теоретические}: Интеграция результата универсальности SRL Sato--Kanazawa как микроструктурного обоснования финансовой LRD
    
    \item \textbf{Практические}: Рекомендации для систем ИИ, работающих на рынках с фрактальной памятью
\end{enumerate}

Более широкая импликация состоит в том, что сложные системы --- будь то рынки, геофизические процессы или биологические ритмы --- разделяют универсальные статистические сигнатуры, возникающие из их внутренней организации. Закон квадратного корня иллюстрирует, как универсальные законы физического типа могут возникать даже в системах, управляемых человеческим поведением.

%==============================================================================
\section*{Доступность данных и кода}
%==============================================================================

Все источники данных, код анализа и материалы воспроизводимости архивированы:

\begin{center}
\textbf{Zenodo DOI}: \url{https://doi.org/10.5281/zenodo.17993213}\\
\textbf{GitHub}: \url{https://github.com/Muhomor2/LRD-v6-Empirical-Evidence}
\end{center}

%==============================================================================
% REFERENCES
%==============================================================================

\begin{thebibliography}{99}

\bibitem{sato2025prl}
Y.~Sato and K.~Kanazawa,
``Strict universality of the square-root law in price impact across stocks: A complete survey of the Tokyo stock exchange,''
\textit{Phys. Rev. Lett.} \textbf{135}, 257401 (2025).

\bibitem{sato2024arxiv}
Y.~Sato and K.~Kanazawa,
``Does the square-root price impact law belong to the strict universal scalings?,''
arXiv:2411.13965 (2024).

\bibitem{bouchaud2025viewpoint}
J.-P.~Bouchaud,
``The Universal Law Behind Market Price Swings,''
\textit{APS Physics} Viewpoint (2025).

\bibitem{bouchaud2018book}
J.-P.~Bouchaud, J.~Bonart, J.~Donier, and M.~Gould,
\textit{Trades, Quotes and Prices: Financial Markets Under the Microscope}
(Cambridge University Press, 2018).

\bibitem{lillo2005theory}
F.~Lillo, S.~Mike, and J.~D.~Farmer,
``Theory for long memory in supply and demand,''
\textit{Phys. Rev. E} \textbf{71}, 066122 (2005).

\bibitem{mandelbrot1968}
B.~B.~Mandelbrot and J.~W.~Van Ness,
``Fractional Brownian motions, fractional noises and applications,''
\textit{SIAM Rev.} \textbf{10}, 422 (1968).

\bibitem{peng1994dfa}
C.-K.~Peng et al.,
``Mosaic organization of DNA nucleotides,''
\textit{Phys. Rev. E} \textbf{49}, 1685 (1994).

\bibitem{peng1992dna}
C.-K.~Peng et al.,
``Long-range correlations in nucleotide sequences,''
\textit{Nature} \textbf{356}, 168 (1992).

\bibitem{peng1995hrv}
C.-K.~Peng, S.~Havlin, H.~E.~Stanley, and A.~L.~Goldberger,
``Quantification of scaling exponents and crossover phenomena in nonstationary heartbeat time series,''
\textit{Chaos} \textbf{5}, 82 (1995).

\end{thebibliography}

\end{document}
